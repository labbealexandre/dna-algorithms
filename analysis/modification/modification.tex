\documentclass{article}
\usepackage[utf8]{inputenc}
\usepackage{amsmath}

\begin{document}

\section{Sparse graph - bound degree : $12\Delta_{avg}$ to $6\Delta_{avg}$}

\subsection{Bound degree of the low degree nodes}

For each low degree node i, let's denote :
\begin{itemize}
    \item $\alpha_i$ its degree
    \item $\beta_i$ the number of edges this low degree node can help
    \item $\gamma_i$ its final degree
\end{itemize}
The following condition holds: $$\gamma_i \leq 3(2\beta_i + \alpha_i)$$

We have to choose $\beta_i$, ie how many edges each low degree node
can help. Let note B the number of edges we have to help.
So we have to choose $\beta_i$ in order to have $$B \leq \sum_{i} \beta_i$$

\subsubsection{In the paper, we chose $\beta_i$ = $\Delta_{avg}$}

So $$\sum_{i} \beta_i = \frac{n*\Delta_{avg}}{2} = m \geq B$$
And we have $\gamma_i \leq 12\Delta_{avg}$ because $\alpha_i \leq 2\Delta_{avg}$

\subsubsection{But we can choose a smaller value : $\beta_i$ = $\Delta_{avg}$ - $\frac{\alpha_i}{2}$}

Let note A the number of edges in which a low degree node is involved.
We have $$A \leq \sum_{i} \alpha_i \leq 2A$$
So $$\sum_{i} \beta_i = \frac{n*\Delta_{avg}}{2} - \frac{1}{2}\sum_{i} \alpha_i \geq m - A$$
But we can also say that $ A + B \leq m$ because A corresponds to edges
into at least one low degree node is involved and B corresponds to edges
into none is involved. So we have $$\sum_{i} \beta_i \geq B$$
And we have too
$$\gamma_i \leq 3(2*\Delta_{avg}-2*\frac{\alpha_i}{2}+\alpha_i) = 6\Delta_{avg}$$

\subsection{Bound degree of the high degree nodes which are neither high-in nor high-out degree}

Let's call strict High degree node an high degree node which is neither
high-in nor high-out degree. According to the paper, they keep their
degree from the initial distribution, which is at most $4\Delta_{avg}$.
But these neighbours can be high-out or high-in degree nodes.*
So we can imagine a case even worst where a strict high degree node
is involved into $4\Delta_{avg}$ Melhorn trees which makes its
final degree at $12\Delta_{avg}$.\\

So let's do a second modification to keep this bounded degree of $6\Delta_{avg}$.
Originally we changed edges only from High-out degree nodes
to High-in degree nodes. we can extend the rule that way :
We now change all edges between high degree nodes.\\

The final degree of a strict high degree node is at most $4\Delta_{avg}$.
Because it is not connected to any high-in or high-out degree nodes.
By doing so there are more edges to replace.
But still we have $ A + B <= m$
because the edges to replace are between High degree nodes.\\
% Also this doesn't change the final degree of high-in or
% high-out degree nodes which is at most $6\Delta_{avg} + 1$.

It also diminishes a lot the max degree of High out(in) degree nodes.
Because that way it will not be in any melhorn tree.
If the node is both high out and high in degree, it's final degree will
be 2. Otherwise it will be inferior to $1+2\Delta_{avg}$.

\newpage
\section{Sparse graph : try to get a constant bound}

\subsection{Introduce a new parameter $c$ to have $\beta_i=c-\frac{\alpha_i}{2}$}

The idea is to classify the degrees in another way and then to do a
quite same reasoning. Let's note $C$ the constant degree we wish to
obtain as a degree bound. We note: $c = \frac{C}{6}$ and $n_l = \frac{m}{c}$.
$n_l$ corresponds to the number of low degrees. In the previous version
we had chosen $n_l = \frac{m}{\Delta_{avg}} = \frac{m}{\frac{2m}{n}} = \frac{n}{2}$
which is the half.\\

The other nodes are high degree nodes and we also change the definition
of high-out(in) degree nodes. These are the nodes with out(in)
degree $> 2c$.\\

Let's do the first step, we replace all the edges between high degree nodes.
There are B edges to replace.We now decide to choose $\beta_i$
that way: $$\beta_i = c - \frac{\alpha_i}{2}$$
Before thanks to pigeon holes principle we were sure that $\beta_i$
was positive. But now we have not this guarantee. To be sure that
$\beta_i$ will always be positive we need to study the worst case.\\

Let's sort the degrees. [i, ..., nl] are the low degree nodes and
[nl+1, ..., n] are the high ones. We take a look at the total degree:
$$n\Delta_{avg} = \sum_{i < nl} \alpha_i + \alpha_{nl} + \sum_{i > nl} \alpha_i$$
$$n\Delta_{avg} >= \alpha_{nl} + \sum_{i > nl} \alpha_i$$
and if $i > nl$ then $\alpha_i > \alpha_nl$ because the nodes are sorted. So
$$n\Delta_{avg} >= \alpha_{nl}(n-nl+1)$$
$$\alpha_{nl} <= \frac{n\Delta_{avg}}{n-nl+1} <= \frac{n\Delta_{avg}}{n-nl}$$
And all the low degree nodes have this bound.
Let's find a condition on c to have $\beta_i >= 0$:
$$c - \frac{n\Delta_{avg}}{2(n-nl)} >= 0$$
$$2(n-\frac{m}{c})*c >= n\Delta_{avg}$$
$$c >= \frac{n\Delta_{avg}+2m}{2n}$$
$$c >= \frac{2m}{n} = \Delta_{avg}$$
So we didn't found a better bound than previously...
We could do the same reasoning but we would have a bound of $6c > 6\Delta_{avg}$.

\subsection{Introduce a second parameter $d$ to have $\beta_i=c-\frac{\alpha_i}{d}$}

Let's find a sufficient condition on d to have $\sum_{i} \beta_i >= B$
$$\sum_{i} \beta_i = m - \frac{1}{d}\sum_{i} \alpha_i$$
So $d >= 2$ is sufficient because we still have $\sum_{i} \alpha_i <= 2A$.\\

Now we also need sufficient conditions on d and c to have $\beta_i >= 0$
Let's note $k = \frac{n_l}{n}$ which is the fraction of low degree nodes.
We have $$c = \frac{\Delta_{avg}}{2k}$$
and $$\alpha_i <= \frac{n\Delta_{avg}}{n-n_l} = \frac{\Delta_{avg}}{1-k}$$
So $$\beta_i >= c - \frac{\Delta_{avg}}{d(1-k)}$$

So by writing equivalent inequalities :
$$c - \frac{\Delta_{avg}}{d(1-k)} >= 0$$
$$\frac{\Delta_{avg}}{2k} >= \frac{\Delta_{avg}}{d(1-k)}$$
$$2k <= d(1-k)$$
$$d >= \frac{2k}{1-k}$$
So this is a sufficient condition to have $\beta_i >= 0$\\

Now let's calculate the bound on low degree node :
$$\gamma_i <= 3(2\beta_i+\alpha_i)$$
$$\gamma_i <= 3(2c - 2\frac{\alpha_i}{d} + \alpha_i)$$
$$\gamma_i <= \frac{3\Delta_{avg}}{k} + 3\alpha_i\frac{d-2}{d}$$
And because $\alpha_i <= \frac{\Delta_{avg}}{1-k}$
$$\gamma_i <= 3\Delta_{avg}\left(\frac{1}{k}+\frac{1-\frac{2}{d}}{1-k}\right)$$
Let's note $$g(k,d) = \frac{1-\frac{2}{d}}{1-k}$$
We need to find a solution of the following system:

\begin{equation*}
\begin{cases}
  \frac{1}{k} + g(k,d) < 2\\
  d >= \frac{2k}{1-k}
\end{cases}
\end{equation*}

On one hand, because we have $d > 2$, we have $g(k, d) > 0$ So it means that we need
to have $k > 0.5$ otherwise $\frac{1}{k} >= 2$.\\

On the other hand, thanks to the first inequation we have :
$$\frac{1-k}{2k} + \frac{1}{2} - \frac{1}{d} < 1-k$$
And because of the second one we have 
$$\frac{1-k}{2k} >= \frac{1}{d}$$
So $$\frac{1}{d} + \frac{1}{2} - \frac{1}{d} < 1 - k$$
Finaly $$k < 0.5$$

So unfortunately, there is no solution. We need to cut in half.

Then $$\gamma_i <= 12\Delta_{avg}\left(1-\frac{1}{d}\right)$$
And if we want this bound to be lower than $6\Delta_{avg}$ then $d = 2$
So it corresponds to the initial definition of $\beta_i$ 

\newpage

\section{Use of high degree nodes to help}

Let's drop the definition of $\beta_i$ and consider the problem in another
way. Recall that we have to help enough edges (B), and if we are able to help
$m - \frac{1}{2}\sum_{i \in L}\alpha_i$ with L the set of low degree nodes,
it is sufficient.\\

Without helping edges, a low degree node will have a final degree lower
than $\gamma_i <= 3*\alpha_i$. Let's try to optimize the bound of the
total degree of low degree nodes M. we have a budget of
$$M - \sum_{i \in L}3*\alpha_i$$
degrees left.
Helping an edge has a cost of 6 (at most 2 melhorn trees)
So we can help $$\frac{M}{6} - \frac{1}{2}\sum_{i \in L}\alpha_i$$
edges.
So let's try to find M to have
$$\frac{M}{6} - \frac{1}{2}\sum_{i \in L}\alpha_i = m - \frac{1}{2}\sum_{i \in L}\alpha_i$$
It leads to
$$M = 6m$$
and so if we distribute the charge equally then the bound of a low degree node
will be $$\gamma_i <= \frac{6m}{\frac{n}{2}} = 6\Delta_{avg}$$

So we didn't manage to reduce the bound unfortunately. But we can add a
new modification. The idea is to use some of the high degree nodes to help too. It enables
us to increase our budget. Now we note M in order to have $\frac{M}{2}$
to be the bound of the total degree of low degree nodes and also for high
degree nodes. That's way M is the bound of the total degree.

Our budget is now :
$$M - 3\sum_{i \in L}\alpha_i - \sum_{i \in H}\alpha_i$$
We have $\sum_{i \in H}\alpha_i <= 2m$
So our budget is greater than
$$M - 3\sum_{i \in L}\alpha_i -2m$$
which enables us to help
$$\frac{M}{6} - \frac{1}{2}\sum_{i \in L}\alpha_i - \frac{m}{3}$$
edges. Again let's try to find the best M.
$$\frac{M}{6} - \frac{1}{2}\sum_{i \in L}\alpha_i - \frac{m}{3} = m - \frac{1}{2}\sum_{i \in L}\alpha_i$$
$$M = 8m$$
And then by distributing equally the charge, we have a degree bound of
$$\frac{M}{n} = \frac{4*2m}{n} = 4\Delta_{avg}$$

\end{document}
