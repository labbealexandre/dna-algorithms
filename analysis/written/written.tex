\documentclass{article}
\usepackage[utf8]{inputenc}
\usepackage{amsmath}
\usepackage{amsfonts}

\begin{document}

\section{Maximum degree of $4\Delta_{avg} + 5$}

\subsection{Algorithm}

For each node i, let's denote $\alpha_i$ its initial degree before running
the algorithm.
We separate the nodes in two subsets $L$ and $H$ which respectively contain
low-degree nodes and the high-degree nodes. The low-degree nodes are
the $\frac{n}{4}$ nodes with the lowest degree, and the high-degree nodes
are the other ones.
Amongst the high-degree nodes, we call high-out-degree (resp. high-in-degree)
an high-degree node whose out-degree (resp. in-degree) is greater than
$2\Delta_{avg}$.
We note HO and HI the subsets of high-out and high-in-degree nodes.
Note that high-degree nodes can be neither high-out nor high-in-degree.\\\\
We want to remove all the initial connexions between high-degree nodes.
To remove an edge, we use the help of another node which will reroute
it, by adding a first edge from the source to him and a second
one from him to the destination.
Both low and high-degree nodes can help, and we note
$\beta_i$ the number of edges a node i can help.\\\\
Let's choose $\beta_i$ that way:
\begin{itemize}
  \item if $i \in L$, then
  $\beta_i = \lceil\frac{2}{3}\Delta_{avg}-\frac{1}{2}\alpha_i\rceil \in \mathbb{N}$
  \item if $i \in H$, then
  $\beta_i = max(\lceil\frac{2}{3}\Delta_{avg}-\frac{1}{6}\alpha_i\rceil, 0) \in \mathbb{N}$
\end{itemize}
Thanks to helping nodes, we remove all the initial connexions between high-degree nodes.\\\\
Before going further in the algorithm, let's justify that for a low-degree
node i, $\beta_i$ is indeed positive because it may not be intuitive.\\\\
We can show it by doing a observation on low-degree nodes.
Let's study a general case where the low-degree nodes are the
$n_l$ lowest ones and the high-degrees nodes the other ones ($0 < n_l < n$).
Let's sort the degrees. [i, ..., nl] are the low-degree nodes and
[nl+1, ..., n] are the high-degree ones. We take a look at the total degree:
$$n\Delta_{avg} = \sum_{i < nl} \alpha_i + \alpha_{nl} + \sum_{i > nl} \alpha_i$$
$$n\Delta_{avg} \geq \alpha_{nl} + \sum_{i > nl} \alpha_i$$
and if $i > nl$ then $\alpha_i > \alpha_nl$ because the nodes are sorted. So
$$n\Delta_{avg} \geq \alpha_{nl}(n-nl+1)$$
$$\forall i \in L, \alpha_i \leq \alpha_{nl} \leq \frac{n\Delta_{avg}}{n-nl+1} \leq \frac{n\Delta_{avg}}{n-nl}$$
By replacing $n_l$ with $\frac{n}{4}$, we have
$\forall i \in L, \alpha_i \leq \frac{4}{3}\Delta_{avg}$ and so $\beta_i >= 0$\\\\
Once we have removed the initial edges between high-degree nodes,
we do melhorn trees with the incoming edges of high-in-degree
nodes and with outgoing edges of high-out-degree nodes.
Some high-in and high-out-degree nodes may have helped nodes.
In that case we do not include those helped nodes in the melhorn tree of the helper.

\subsection{Analysis}

The first thing to know is if we are able, with our defintion of $beta_i$
to help enough edges. Let's note B the number of edges to remove and
A the number of edges in which a low degree node is involved.
B exactly equals to the number of initial edges between high-degree
nodes because those are the edges we want to remove. So we have :
$$m = A + B$$
Furthermore we have
$$\sum_{i \in L}\alpha_i \leq 2A$$
Because for an extreme case, there is no edges between low
and high-degree nodes so $\sum_{i \in L}\alpha_i = 2A$.
Otherwise we have connexions with high-degree nodes which will
increase A for a given total degree $\sum_{i \in L}\alpha_i$.
Then it leads to
$$B \leq m - \frac{1}{2}\sum_{i \in L}\alpha_i$$
It means that beeing able to help more than $m - \frac{1}{2}\sum_{i \in L}\alpha_i$
is sufficient to remove all the initial edges between high-degree nodes.
Now we show that our choice of $\beta_i$ allows us to help a sufficient number
of edges. To this end we sum the $\beta_i$.
\begin{itemize}
  \item if $i \in L$,
  $\beta_i \geq \frac{2}{3}\Delta_{avg}-\frac{1}{2}\alpha_i$
  $$\sum_{i \in L}\beta_i \geq \frac{2}{3}*\frac{n}{4}\Delta_{avg} - \frac{1}{2}\sum_{i \in L}\alpha_i$$
  $$\sum_{i \in L}\beta_i \geq \frac{m}{3} - \frac{1}{2}\sum_{i \in L}\alpha_i$$
  \item if $i \in H$,
  $\beta_i \geq \frac{2}{3}\Delta_{avg}-\frac{1}{6}\alpha_i$
  $$\sum_{i \in L}\beta_i \geq \frac{2}{3}*\frac{3n}{4}\Delta_{avg} - \frac{1}{6}\sum_{i \in H}\alpha_i$$
  $$\sum_{i \in L}\beta_i \geq m - \frac{1}{6}\sum_{i \in H}\alpha_i$$
  And because $\sum_{i \in H}\alpha_i \leq 2m$
  We have
  $$\sum_{i \in L}\beta_i \geq m - \frac{1}{3}m$$
\end{itemize}
We sum our two previous results in order to have a lower bound on
$\sum_{i}\beta_i$
$$\sum_{i}\beta_i \geq m - \frac{1}{2}\sum_{i \in L}\alpha_i$$
And that way we are able to help enough edges.\\\\
Finally we evaluate the maximal final degree of the nodes. We note
$\gamma_i$ the final degree of the node i. We study the different
subsets of nodes to find an upper bound for each case. For any node,
to be involved in a melhorn tree can increase a lot its initial degree.
For one initial connexion which will be replaced in a melhorn tree,
it can lead at the end to three connexions.

\begin{itemize}
  \item if $i \in L$, in the worst case, the initial edges from i are
  going to high-in degree node and the initial edges in destination to i
  are coming from high-out degree node. So for each initial edge, $i$ will
  be involved in a melhorn tree. So the degree increases from $\alpha_i$
  to at most $3\alpha_i$. But this node may also have helped edges and
  in the worst case, it has only helped edges from high-out to
  high-in-degree nodes.
  This adds $3 \times 2\beta_i$ more to the final degree. At the end we have
  $$\gamma_i \leq 3(2\beta_i +\alpha_i)$$
  and
  $$\beta_i < \frac{2}{3}\Delta_{avg} - \frac{1}{2}\alpha_i + 1$$
  So
  $$\gamma_i \leq 4\Delta_{avg} + 5$$
  \item if $i \in HO \cap HI$, the intial connexions of i will not lead
  to melhorn trees because they have been replaced by connexions
  with an helper node. And because we forced it in the description
  of the algorithm, if this helper creates a melhorn tree, $i$ will
  not be involved in. So the degree of $i$ due to initial connexions
  will not increase. Even better because $i$ is both high-out and
  high-in degree, and making so two melhorn trees on intial in and out
  edges, this degree will decrease to 2. Like before we need to add
  the contribution in the degree of helping other nodes.
  $$\gamma_i \leq 2 + 6\beta_i$$
  if $\beta_i = 0$, $\gamma_i \leq 2$
  else $\beta_i = \lceil\frac{2}{3}\Delta_{avg}-\frac{1}{6}\alpha_i\rceil > 0$
  $$\gamma_i < 2 + 6(\frac{2}{3}\Delta_{avg} - \frac{1}{6}\alpha_i + 1) = 8 + 4\Delta_{avg} - \alpha_i$$
  But $\alpha_i > 4\Delta_{avg}$ so $\gamma_i \leq 7$
  \item if $i \in HO \setminus HI$ (same reasoning for $i \in HI \setminus HO$)\\\\
  Let's note $\alpha_i^+$ the initial out-degree of i and $\alpha_i^-$ the initial in-degree
  We have :
  \begin{enumerate}
    \item $\alpha_i = \alpha_i^+ + \alpha_i^-$
    \item $\alpha_i^+ \geq 2\Delta_{avg}$ because $i \in HO$
    \item $\alpha_i^- < 2\Delta_{avg}$ because $i \notin HI$
  \end{enumerate}
  By doing the same reasoning as previously, the contribution of
  initial out-edges will be 1, because we do a melhorn from them
  and the contribution of intial in-edges will be $\alpha_i^-$.
  Again we add the contribution of helping other nodes.
  $$\gamma_i \leq 1 + \alpha_i^- + 6\beta_i$$
  if $\beta_i = 0$, $\gamma_i \leq 2 + \alpha_i^- < 2 + 2\Delta_{avg}$
  else
  $$\gamma_i < 1 + \alpha_i^- + 6(\frac{2}{3}\Delta_{avg} - \frac{1}{6}\alpha_i + 1) = 8 + 4\Delta_{avg} - \alpha_i^+$$
  But $\alpha_i^+ > 2\Delta_{avg}$
  so $\gamma_i \leq 6 + 2\Delta_{avg}$
  \item if $i \in H \setminus (HI \cup HO)$, their will be no melhorn
  trees from the initial connexions because the out and in-degrees
  are less than $2\Delta_{avg}$. So the contribution of initial
  connexion will be equals to $\alpha_i$. One last time we add the
  contribution of helping nodes.
  $$\gamma_i \leq \alpha_i + 6\beta_i$$
  because $\alpha_i < 4\Delta_i$, $\beta_i = \lceil\frac{2}{3}\Delta_{avg}-\frac{1}{6}\alpha_i\rceil > 0$
  so $$\gamma_i < \alpha_i + 6(\frac{2}{3}\Delta_{avg}-\frac{1}{6}\alpha_i + 1)$$
  $$\gamma_i \leq 4\Delta_{avg} + 5$$
\end{itemize}
Finally if we sum up these results, we can say that this algorithm will
generate a DNA N with maximum degree of $4\Delta_{avg} + 5$ instead of
$12\Delta_{avg}$ in the previous version.

\end{document}
