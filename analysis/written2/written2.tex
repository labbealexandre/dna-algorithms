\documentclass{article}
\usepackage[utf8]{inputenc}
\usepackage{amsmath}
\usepackage{amsfonts}

\begin{document}

\section{Maximum degree of $3\Delta_{avg} + 8$}

Maciek: Intro to the algorithm, comparison with DISC algorithm.
(high degree nodes participate as helpers, H-H edges are replaced, low degree nodes are $n/4$ of all nodes)

\subsection{Algorithm}

The algorithm replaces all the initial edges with 2-hop paths through an intermediate node.
We say that an intermediate node \emph{helps} (is a \emph{helper}) an edge.
We say that the edges added to (and from) intermediate nodes are \emph{intermediate edges}.
For a node $i$, we denote $\beta_i$ as the limit on number of edges the node $i$ can help.

We choose the values $\beta_i$ in the following way: $\beta_i = \lceil\Delta_{avg}/2\rceil$

\medskip

Note that we reduce the number of the new connections, and then the 
final degree, if for an edge to replace we choose one of the ends as
the helper. It would be a better option for a concrete implementation than
an arbitrary choice of the helpers. However the analysis stands for the
general case.

\medskip

After this procedure, a node $i$ has two types of new neighbors, 
a set $G_i$ of intermediate nodes that have rerouted an inital edge
between $i$ and another node, and a set $H_i$ of nodes initially
connected in pairs that $i$ has rerouted. For each node $i$, the algorithm constructs two separate Mehlhorn trees:
an ingoing one and an outgoing one. It forms Mehlhorn trees out of $G_i$.

\medskip

Now we claim that the algorithm has a sufficient number of edges available as helpers
 (ie., ths sum of $\beta_i$ of all nodes is sufficient).

  $$\sum_i \beta_i >= \sum_i \Delta_{avg}/2 = \frac{n\Delta_{avg}}{2} = m$$

It can help more than the total number of edges $m$, and thus it is well-defined.

\medskip

Finally, we evaluate the maximal final degree of the nodes.
Let $\gamma_i$ be the final degree of the node $i$.
The degree of each node may increase beyond its initial degree because it helps other nodes, and, more importantly, because it may participate in Mehlhorn trees of its neighbors.

\medskip

A node $i$ is not involved in the Melhorn trees of the intermediate nodes
that have rerouted a node between $i$ and another node. Furthermore
these new connections with intermediate nodes are packed in 2 Mehlhorn trees
whose root is $i$. Additionally $i$ is involved in one Melhorn tree
for each node it helped, in total at most $2\beta_i$ trees.
A participation in each Mehlhorn tree adds at most $3$ edges to a node, thus

$$\gamma_i \leq 2 + 6\beta_i \leq 2 + 6 \times \left(\frac{\Delta_{avg}}{2}+1\right) = 8 + 3\Delta_{avg},$$

It guarantees that the algorithm produces a network
with maximum degree of $3\Delta_{avg} + 8$.

\end{document}
