\documentclass{article}
\usepackage[utf8]{inputenc}
\usepackage{amsmath}
\usepackage{amsfonts}

\begin{document}

\section{Maximum degree of $4\Delta_{avg} + 7$}

Maciek: Intro to the algorithm, comparison with DISC algorithm.
(high degree nodes participate as helpers, H-H edges are replaced, low degree nodes are $n/4$ of all nodes)

\subsection{Algorithm}

The algorithm partitions the nodes into two subsets: $L$, called the low degree nodes and $H$, called
the high degree nodes. The low-degree nodes are
the $\frac{n}{4}$ nodes with the minimal degree, and the high degree nodes
are the remaining ones.
Among the high degree nodes, we distinguish the set $HO$ (resp. $HI$), called the high-out degree nodes (resp. high-in degree) that contain nodes whose out-degree (resp. in-degree) is greater than
$2\Delta_{avg}$.
Note that high degree nodes can be neither high-out nor high-in degree.

The algorithm replaces all the initial edges between high degree nodes with 2-hop paths through an intermediate node.
We say that an intermediate node \emph{helps} (is a \emph{helper}) an edge between two high degree nodes.
We say that the edges added to (and from) intermediate nodes are \emph{intermediate edges}.
The algorithm uses both low and high degree nodes as helpers.
For a node $i$, we denote $\beta_i$ as the limit on number of edges the node $i$ can help.
We denote $\alpha_i$ the initial degree of node $i$ (prior to the execution of the algorithm).

We choose the values $\beta_i$ in the following way:
\begin{itemize}
  \item for $i \in L$ we set
  $\beta_i = \lceil\frac{2}{3}\Delta_{avg}-\frac{1}{2}\alpha_i\rceil$
  \item for $i \in H$ we set
  $\beta_i = max(\lceil\frac{2}{3}\Delta_{avg}-\frac{1}{6}\alpha_i\rceil, 0)$
\end{itemize}
We'll later justify that these values are positive.

The algorithm reduces the degree of $HI$ (resp. $HO$) nodes by forming a~Mehlhorn tree out of its ingoing (resp. outgoing) neighbors.
For nodes that are both high in-degree and high out-degree we construct two separate Mehlhorn trees: an ingoing one and an outgoing one.
It forms Mehlhorn trees out of both initial and intermediate edges.


High-in and high-out degree nodes may help other nodes (that are also $H$).
%Then, an intermediate edge is constructed, and we skip this node while building a Mehlhorn tree of its neighbors ().
When a high-out degree node $i$ helps a high degree node $j$, we skip $j$ while building an outgoing Mehlhorn tree of $i$ (and similarly if $i$ is high-in degree node).
This is done for two reasons. First, this would be unnecessary because they are already connected directly, but second, we don't want a high degree node involved in too many Mehlhorn trees of its neighbors.


\medskip

Now we argue that our choice of $\beta_i$ is positive for low degree nodes.
To this end, we sort the the nodes in ascending degree order, and then the first
$\langle 1, \ldots, n/4\rangle$ nodes are the low-degree ones and
the last $\langle n/4 +1, \ldots, n\rangle$ nodes are the high-degree ones.
The sum of degrees of all nodes is then

$$n\Delta_{avg} = \sum_{i < n/4} \alpha_i + \alpha_{n/4} + \sum_{i > n/4} \alpha_i \geq \alpha_{n/4} + \sum_{i > n/4} \alpha_i \geq \alpha_{n/4}(n-n/4+1),$$
where the last inequality follows as 
for all $i > n/4$ we have $\alpha_i > \alpha_{n/4}$ (the nodes are sorted).
Than for all $i \in L$
$$\alpha_i \leq \alpha_{n/4} \leq \frac{n\Delta_{avg}}{n-n/4+1} \leq \frac{n\Delta_{avg}}{n-n/4} \leq \frac{4}{3}\Delta_{avg},$$
and thus for all $i \in L$ we have $\beta_i \geq 0$.

\medskip

Now we claim that we have sufficient number of edges available as helpers
 (ie., ths sum of $\beta_i$ of all nodes is sufficient).
%For the algorithm to be well-defined, we must have the sufficient
%number of helping nodes to help all edges between high degree nodes,
%i.e., 
Let $B$ be the number of initial edges between high degree nodes.
We need a helper for each of them, thus
\begin{equation}
  \label{eq:betaB}
 \sum_i \beta_i \geq B
\end{equation}
Recall that our algorithm helps only high degree nodes.
Let $A$ be the number of initial edges in which a low degree node is involved.
In total, these sum to all edges,
$$m = A + B\enspace.$$
%
Now we bound the number of edges $B$ the algorithm helps.
Each edge counted in $A$ involves at most $2$ low degree nodes, thus
$\sum_{i \in L}\alpha_i \leq 2A$.
This gives us
$$B \leq m - \frac{1}{2}\sum_{i \in L}\alpha_i\enspace.$$
%
Now we show that our choice of $\beta_i$ satisfies the condition \eqref{eq:betaB}.
For low degree nodes $L$ we have   $\beta_i \geq \frac{2}{3}\Delta_{avg}-\frac{1}{2}\alpha_i$
and thus

  $$\sum_{i \in L}\beta_i \geq \frac{2}{3}\cdot\frac{n}{4}\Delta_{avg} - \frac{1}{2}\sum_{i \in L}\alpha_i \geq \frac{m}{3} - \frac{1}{2}\sum_{i \in L}\alpha_i$$
  For high degree nodes $H$ we have 
  $\beta_i \geq \frac{2}{3}\Delta_{avg}-\frac{1}{6}\alpha_i$
  and thus
  $$\sum_{i \in H}\beta_i \geq \frac{2}{3}\cdot\frac{3n}{4}\Delta_{avg} - \frac{1}{6}\sum_{i \in H}\alpha_i \geq m - \frac{1}{6}\sum_{i \in H}\alpha_i \geq m - \frac{1}{3}m,$$
  where the last inequality follows from $\sum_{i \in H}\alpha_i \leq 2m$.
Finally, we obtain a lower bound on
$\sum_{i}\beta_i$
$$\sum_{i}\beta_i \geq m - \frac{1}{2}\sum_{i \in L}\alpha_i \geq B,$$
and we conclude that the algorithm has a sufficient number of
helping edges, and thus it is well-defined.

\medskip

Finally, we evaluate the maximal final degree of the nodes.
Let $\gamma_i$ be the final degree of the node $i$.
The degree of each node may increase beyond its initial degree because it helps other nodes, and, more importantly, because it may participate in Mehlhorn trees of its neighbors.

\medskip

A low degree node may be involved in Mehlhorn trees of all its neighbors (both initial and intermediate). 
The total number of the node's neighbors is the number of its initial neighbors $\alpha_i$
plus at most two high degree nodes per each edge it helps, in total $\alpha_i + 2\beta_i$.
A participation in each Mehlhorn tree adds at most $3$ edges to a node, thus

  $$\gamma_i \leq 3(2\beta_i +\alpha_i) \leq 4\Delta_{avg} + 6,$$
  where the last inequality follows from 
  $\beta_i = \lceil\frac{2}{3}\Delta_{avg}-\frac{1}{2}\alpha_i\rceil < \frac{2}{3}\Delta_{avg} - \frac{1}{2}\alpha_i + 1$.

\medskip

A node $i$ that is both high in and high out degree (i.e., $i \in HO \cap HI$),
becomes the root of two Mehlhorn trees. 
The algorithm never involves such node in helper's Mehlhorn tree.
All of $i$'s initial edges are replaced with two edges that are leading to Mehlhorn trees that contain all of its neighbors (and possibly some intermediate nodes).
The node $i$ cannot help other nodes because $\alpha_i \geq 4\Delta_{avg}$, and thus $\beta_i = 0$, and finally $\gamma_i = 2$.

\medskip

For $i \in HO \setminus HI$ (and similarly for $i \in HI \setminus HO$),
we partition its initial degree $\alpha_i$ into initial out degree $\alpha_i^+$ and
initial in degree $\alpha_i^-$. 
From $i \in HO \setminus HI$, $\alpha_i^+ \geq 2\Delta_{avg}$ and $\alpha_i^- > 2\Delta_{avg}$.
All initial outgoing edges are replaced with Mehlhorn trees, so $i$'s out-degree is $1$ (from connection to the Mehlhorn tree) plus $6$ per each edge it is helping (thus it is involved in at most $2$ Mehlhorn trees, each adding at most $3$ edges).
Additionally, its in-going degree is $\alpha_i^-$, as each of its neighbors is either directly connected with it, or this connection is replaced by an intermediate node (that exchanges one edge for one edge).
In total, the final degree of $i$ is
$$\gamma_i \leq 1 + \alpha_i^- + 6\beta_i$$
if $\beta_i = 0$, $\gamma_i \leq 2 + \alpha_i^- < 2 + 2\Delta_{avg}$
else
$$\gamma_i < 1 + \alpha_i^- + 6(\frac{2}{3}\Delta_{avg} - \frac{1}{6}\alpha_i + 1) =  4\Delta_{avg} - \alpha_i^+ + 7 < 2\Delta_{avg}+7,$$
where the last inequality follows from $\alpha_i^+ > 2\Delta_{avg}$.

A node $i$ that is neither high in nor high out degree (i.e., $i \in H \setminus (HI \cup HO)$),
will not be involved in the Mehlhorn trees of its initial neighbors. The algorithm
replaced its initial connections with high degree nodes using 
intermediate nodes. And recall that helpers do not involve
the nodes they helped in their Mehlhorn trees.
The final degree of $i$ is
$$\gamma_i \leq \alpha_i + 6\beta_i < \alpha_i + 6(\frac{2}{3}\Delta_{avg}-\frac{1}{6}\alpha_i + 1) < 4\Delta_{avg} + 6,$$
where the second inequality follows from
$\alpha_i < 4\Delta_i$ and $\beta_i = \lceil\frac{2}{3}\Delta_{avg}-\frac{1}{6}\alpha_i\rceil > 0$.

All of the above bounds combined guarantee that the algorithm produces a network
with maximum degree of $4\Delta_{avg} + 7$.

\end{document}
