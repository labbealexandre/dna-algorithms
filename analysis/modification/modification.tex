\documentclass{article}
\usepackage[utf8]{inputenc}

\begin{document}

\section{Modification of the algorithm}

\subsection{Bound degree of the low degree nodes}

For each low degree node i, let's denote :
\begin{itemize}
    \item $\alpha_i$ its degree
    \item $\beta_i$ the number of edges this low degree node can help
    \item $\gamma_i$ its final degree
\end{itemize}
The following condition holds: $$\gamma_i \leq 3(2\beta_i + \alpha_i)$$

We have to choose $\beta_i$, ie how many edges each low degree node
can help. Let note B the number of edges we have to help.
So we have to choose $\beta_i$ in order to have $$B \leq \sum_{i} \beta_i$$

\subsubsection{In the paper, we chose $\beta_i$ = $\Delta_{avg}$}

So $$\sum_{i} \beta_i = \frac{n*\Delta_{avg}}{2} = m \geq B$$
And we have $\gamma_i \leq 12\Delta_{avg}$ because $\alpha_i \leq 2\Delta_{avg}$

\subsubsection{But we can choose a smaller value : $\beta_i$ = $\Delta_{avg}$ - $\frac{\alpha_i}{2}$}

Let note A the number of edges in which a low degree node is involved.
We have $$A \leq \sum_{i} \alpha_i \leq 2A$$
So $$\sum_{i} \beta_i = \frac{n*\Delta_{avg}}{2} - \frac{1}{2}\sum_{i} \alpha_i \geq m - A$$
But we can also say that $ A + B \leq m$ because A corresponds to edges
into at least one low degree node is involved and B corresponds to edges
into none is involved. So we have $$\sum_{i} \beta_i \geq B$$
And we have too
$$\gamma_i \leq 3(2*\Delta_{avg}-2*\frac{\alpha_i}{2}+\alpha_i) = 6\Delta_{avg}$$

\subsection{Bound degree of the high degree nodes which are neither high-in nor high-out degree}

Let's call strict High degree node an high degree node which is neither
high-in nor high-out degree. According to the paper, they keep their
degree from the initial distribution, which is at most $4\Delta_{avg}$.
But these neighbours can be high-out or high-in degree nodes.*
So we can imagine a case even worst where a strict high degree node
is involved into $4\Delta_{avg}$ Melhorn trees which makes its
final degree at $12\Delta_{avg}$.\\

So let's do a second modification to keep this bounded degree of $6\Delta_{avg}$.
Originally we changed edges only from High-out degree nodes
to High-in degree nodes. we can extend the rule that way :
We now change all edges between high degree nodes.\\

The final degree of a strict high degree node is at most $4\Delta_{avg}$.
Because it is not connected to any high-in or high-out degree nodes.
By doing so there are more edges to replace.
But still we have $ A + B \leq m$
because the edges to replace are between High degree nodes.
Also this doesn't change the final degree of high-in or
high-out degree nodes which is at most $6\Delta_{avg} + 1$.

\end{document}
