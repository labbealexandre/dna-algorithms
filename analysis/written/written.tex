\documentclass{article}
\usepackage[utf8]{inputenc}
\usepackage{amsmath}
\usepackage{amsfonts}

\begin{document}

\section{Maximum degree of $4\Delta_{avg} + 5$}

\subsection{Algorithm}

For each node $i$, let's denote $\alpha_i$ its initial degree prior to the execution of 
the algorithm.
We partition the nodes into two subsets $L$, called the low degree nodes and $H$, called
the high degree nodes. The low-degree nodes are
the $n_L := \frac{n}{4}$ nodes with the lowest degree, and the high degree nodes
are the remaining ones.
Among the high degree nodes, we distinguish the set $HO$ ($HI$), called the high-out degree nodes (resp. high-in degree) that contain nodes whose out-degree (resp. in-degree) is greater than
$2\Delta_{avg}$.
Note that high degree nodes can be neither high-out nor high-in degree.

The algorithm replaces all the initial connections between high degree nodes with 2-hop routes through an intermediate node.
We say that an intermediate node \emph{helps} (i.e., is a \emph{helper}) an edge between two high degree nodes.
We say that the edges added to (and from) intermediate nodes are \emph{intermediate edges}.
The algorithm uses both low and high degree nodes as helpers, and we denote
$\beta_i$ as the limit on number of edges the node $i$ can help.

We choose the values $\beta_i$ in the following way:
\begin{itemize}
  \item for $i \in L$ we set
  $\beta_i = \lceil\frac{2}{3}\Delta_{avg}-\frac{1}{2}\alpha_i\rceil$
  \item for $i \in H$ we set
  $\beta_i = max(\lceil\frac{2}{3}\Delta_{avg}-\frac{1}{6}\alpha_i\rceil, 0)$
\end{itemize}
We'll justify that these values are positive.

Now we reduce the degree of $HI$ (resp. $HO$) nodes by forming a Mehlhorn tree out of its ingoing (resp. outgoing) neigbours.
We form Mehlhorn trees out of both initial and intermediate edges.

\textbf{not precise}


Once we replaced the initial edges between high-degree nodes with intermediate nodes,
we do melhorn trees with the incoming edges of high-in-degree
nodes and with outgoing edges of high-out-degree nodes.
Some high-in and high-out-degree nodes may have helped nodes.
In that case we do not include those helped nodes in the melhorn tree of the helper.


\medskip

Now we argue that our choice of $\beta_i$ is positive for low degree nodes.
Let's sort the the nodes in ascending degree order, so that $\langle 1, \ldots, n/4\rangle$ are the low-degree nodes and
$\langle n/4 +1, \ldots, n\rangle$  are the high-degree ones.
The total degree is then

$$n\Delta_{avg} = \sum_{i < n/4} \alpha_i + \alpha_{n/4} + \sum_{i > n/4} \alpha_i \geq \alpha_{n/4} + \sum_{i > n/4} \alpha_i \geq \alpha_{n/4}(n-n/4+1),$$
where the last inequality follows as 
for all $i > n/4$ we have $\alpha_i > \alpha_{n/4}$ (the nodes are sorted).
Than for all $i \in L$
$$\alpha_i \leq \alpha_{n/4} \leq \frac{n\Delta_{avg}}{n-n/4+1} \leq \frac{n\Delta_{avg}}{n-n/4} \leq \frac{4}{3}\Delta_{avg},$$
and thus for all $i \in L$ we have $\beta_i \geq 0$.

\subsection{Analysis}

Now we claim that we have sufficient number of edges available as helpers ($\sum \beta_i$).
Recall that our algorithm helps only high degree nodes.
Let $B$ be the number of initial edges between high degree nodes, and let $A$ be the number of initial edges in which a low degree node is involved.
In total, these sum to all edges:
$$m = A + B$$

Now we bound the number of edges $B$ the algorithm helps.
Each edge from $A$ (\textbf{A is not a set}) involves at most $2$ low degree nodes, thus
$\sum_{i \in L}\alpha_i \leq 2A$.
This gives us
$$B \leq m - \frac{1}{2}\sum_{i \in L}\alpha_i$$

For the algorithm to be well-defined, we must have the sufficient number of helping nodes to help all edges between high degree nodes, i.e., $\sum_i \beta_i \geq B$.
Now we show that our choice of $\beta_i$ satisfies this condition.
For low degree nodes $L$ we have   $\beta_i \geq \frac{2}{3}\Delta_{avg}-\frac{1}{2}\alpha_i$
and thus

  $$\sum_{i \in L}\beta_i \geq \frac{2}{3}\cdot\frac{n}{4}\Delta_{avg} - \frac{1}{2}\sum_{i \in L}\alpha_i \geq \frac{m}{3} - \frac{1}{2}\sum_{i \in L}\alpha_i$$

For high degree nodes $H$ we have 
  $\beta_i \geq \frac{2}{3}\Delta_{avg}-\frac{1}{6}\alpha_i$
  and thus
  $$\sum_{i \in H}\beta_i \geq \frac{2}{3}\cdot\frac{3n}{4}\Delta_{avg} - \frac{1}{6}\sum_{i \in H}\alpha_i \geq m - \frac{1}{6}\sum_{i \in H}\alpha_i \geq m - \frac{1}{3}m,$$
  where the last inequaliy follows from $\sum_{i \in H}\alpha_i \leq 2m$.
We sum our two previous results in order to obtain a lower bound on
$\sum_{i}\beta_i$
$$\sum_{i}\beta_i \geq m - \frac{1}{2}\sum_{i \in L}\alpha_i \geq B,$$
and we conclude that the algorithm is able to help enough edges, and is well-defined.

\medskip

Finally we evaluate the maximal final degree of the nodes.
Let's denote 
$\gamma_i$ as the final degree of the node $i$.
The degree of each node may increase beyond its initial degree because it helps other nodes, and, more importantly, because it may participate in Mehlhorn trees of its neigbours.

A low degree node may be involved in Mehlhorn trees of all its neighbours (\textbf{both initial and intermediate}). 
The total number of its neighbours is the number of its initial neighbours $\alpha_i$ plus at most two high degree nodes per each edge it helps, in total $\alpha_i + 2\beta_i$.
A participation in each Mehlhorn tree adds at most $3$ edges to a node, thus

  $$\gamma_i \leq 3(2\beta_i +\alpha_i) \leq 4\Delta_{avg} + 5,$$
  where the last inequality follows from 
  $\beta_i = \lceil\frac{2}{3}\Delta_{avg}-\frac{1}{2}\alpha_i\rceil < \frac{2}{3}\Delta_{avg} - \frac{1}{2}\alpha_i + 1$, and the fact that $\gamma_i$ is an integer.


\begin{itemize}
  \item if $i \in L$, in the worst case, the initial edges from i are
  going to high-in degree node and the initial edges in destination to i
  are coming from high-out degree node. So for each initial edge, $i$ will
  be involved in a melhorn tree. So the degree increases from $\alpha_i$
  to at most $3\alpha_i$. But this node may also have helped edges and
  in the worst case, it has only helped edges from high-out to
  high-in-degree nodes.
  This adds $3 \times 2\beta_i$ more to the final degree. At the end we have
  $$\gamma_i \leq 3(2\beta_i +\alpha_i)$$
  and
  $$\beta_i < \frac{2}{3}\Delta_{avg} - \frac{1}{2}\alpha_i + 1$$
  So
  $$\gamma_i \leq 4\Delta_{avg} + 5$$
  \item if $i \in HO \cap HI$, the intial connexions of i will not lead
  to melhorn trees because they have been replaced by connexions
  with an helper node. And because we forced it in the description
  of the algorithm, if this helper creates a melhorn tree, $i$ will
  not be involved in. So the degree of $i$ due to initial connexions
  will not increase. Even better because $i$ is both high-out and
  high-in degree, and making so two melhorn trees on intial in and out
  edges, this degree will decrease to 2. Like before we need to add
  the contribution in the degree of helping other nodes.
  $$\gamma_i \leq 2 + 6\beta_i$$
  if $\beta_i = 0$, $\gamma_i \leq 2$
  else $\beta_i = \lceil\frac{2}{3}\Delta_{avg}-\frac{1}{6}\alpha_i\rceil > 0$
  $$\gamma_i < 2 + 6(\frac{2}{3}\Delta_{avg} - \frac{1}{6}\alpha_i + 1) = 8 + 4\Delta_{avg} - \alpha_i$$
  But $\alpha_i > 4\Delta_{avg}$ so $\gamma_i \leq 7$
  \item if $i \in HO \setminus HI$ (same reasoning for $i \in HI \setminus HO$)\\\\
  Let's note $\alpha_i^+$ the initial out-degree of i and $\alpha_i^-$ the initial in-degree
  We have :
  \begin{enumerate}
    \item $\alpha_i = \alpha_i^+ + \alpha_i^-$
    \item $\alpha_i^+ \geq 2\Delta_{avg}$ because $i \in HO$
    \item $\alpha_i^- < 2\Delta_{avg}$ because $i \notin HI$
  \end{enumerate}
  By doing the same reasoning as previously, the contribution of
  initial out-edges will be 1, because we do a melhorn from them
  and the contribution of intial in-edges will be $\alpha_i^-$.
  Again we add the contribution of helping other nodes.
  $$\gamma_i \leq 1 + \alpha_i^- + 6\beta_i$$
  if $\beta_i = 0$, $\gamma_i \leq 2 + \alpha_i^- < 2 + 2\Delta_{avg}$
  else
  $$\gamma_i < 1 + \alpha_i^- + 6(\frac{2}{3}\Delta_{avg} - \frac{1}{6}\alpha_i + 1) = 8 + 4\Delta_{avg} - \alpha_i^+$$
  But $\alpha_i^+ > 2\Delta_{avg}$
  so $\gamma_i \leq 6 + 2\Delta_{avg}$
  \item if $i \in H \setminus (HI \cup HO)$, their will be no melhorn
  trees from the initial connexions because the out and in-degrees
  are less than $2\Delta_{avg}$. So the contribution of initial
  connexion will be equals to $\alpha_i$. One last time we add the
  contribution of helping nodes.
  $$\gamma_i \leq \alpha_i + 6\beta_i$$
  because $\alpha_i < 4\Delta_i$, $\beta_i = \lceil\frac{2}{3}\Delta_{avg}-\frac{1}{6}\alpha_i\rceil > 0$
  so $$\gamma_i < \alpha_i + 6(\frac{2}{3}\Delta_{avg}-\frac{1}{6}\alpha_i + 1)$$
  $$\gamma_i \leq 4\Delta_{avg} + 5$$
\end{itemize}
Finally if we sum up these results, we can say that this algorithm will
generate a DNA N with maximum degree of $4\Delta_{avg} + 5$ instead of
$12\Delta_{avg}$ in the previous version.

\end{document}
