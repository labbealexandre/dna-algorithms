\documentclass{article}
\usepackage[utf8]{inputenc}
\usepackage{amsmath}
\usepackage{amsfonts}

\begin{document}

\section{Maximum degree of $4\Delta_{avg}$}

\subsection{Algorithm}

For each node i, let's denote :
\begin{itemize}
    \item $\alpha_i$ its intial degree
    \item $\beta_i$ the number of edges this node can help
    \item $\gamma_i$ its final degree
\end{itemize}
We also note $L$ and $H$ the subsets of low-degree and high-degree nodes.
The low degree-nodes are the $\frac{n}{4}$ nodes with the lowest degree.
We note HO and HI the subsets of high-out and high-in-degree nodes.
A node $\in HO$ ($HI$) if its out-degree (in-degree) is greater than
$2\Delta_{avg}$.\\\\
Let's choose $\beta_i$ that way:
\begin{itemize}
  \item if $i \in L$, then
  $\beta_i = \lceil\frac{2}{3}\Delta_{avg}-\frac{1}{2}\alpha_i\rceil \in \mathbb{N}$
  \item if $i \in H$, then
  $\beta_i = max(\lceil\frac{2}{3}\Delta_{avg}-\frac{1}{6}\alpha_i\rceil, 0) \in \mathbb{N}$
\end{itemize}
Thanks to helping nodes, we remove all the initial connexions between high-degree nodes.
Once it is done we do melhorn trees with the incoming edges of an high-in-degree
node and with outgoing edges of an high-out-degree node.
Some high-in and high-out-degree nodes may have helped two high degree nodes.
In that case we do not include these two nodes in the melhorn tree of the helper.

\subsection{Analysis}

Let's first prove that if $i \in L$, $\beta_i \in \mathbb{N}$.
By definition, $\beta_i$ is clearly an integer but we can imagine that
it could be strictly negative. But we have an upper bound on the degree of i
$$\alpha_i <= \frac{n\Delta_{avg}}{n-n_l}$$
with $n_l$ the number of low degree nodes. In this case $n_l = \frac{n}{4}$
so
$$\alpha_i <= \frac{4}{3}\Delta_{avg}$$
And
$$\frac{2}{3}\Delta_{avg}-\frac{1}{2}\alpha_i >= 0$$
So $$\beta_i >= 0$$\\
Let's note B the number of edges to remove and A the number of edges in
which a low degree node is involved. So we have :
$$m = A + B$$
Furthermore we have
$$A <= \sum_{i \in L}\alpha_i <= 2A$$
Because the sum on low-degree nodes is between two extreme cases :
\begin{itemize}
  \item no edges between low degree nodes so $\sum_{i \in L}\alpha_i = A$
  \item no edges between low and high-degree nodes so $\sum_{i \in L}\alpha_i = 2A$
\end{itemize}
Then it leads to
$$B <= m - \frac{1}{2}\sum_{i \in L}\alpha_i$$
It means that beeing able to help more than $m - \frac{1}{2}\sum_{i \in L}\alpha_i$
is sufficient to remove all the initial edges between high-degree nodes.
Let's find a lower bound on $\sum_{i}\beta_i$:
\begin{itemize}
  \item if $i \in L$,
  $$\beta_i >= \frac{2}{3}\Delta_{avg}-\frac{1}{2}\alpha_i$$
  so
  $$\sum_{i \in L}\beta_i >= \frac{2}{3}*\frac{n}{4}\Delta_{avg} - \frac{1}{2}\sum_{i \in L}\alpha_i$$
  $$\sum_{i \in L}\beta_i >= \frac{m}{3} - \frac{1}{2}\sum_{i \in L}\alpha_i$$
  \item if $i \in H$,
  $$\beta_i >= \frac{2}{3}\Delta_{avg}-\frac{1}{6}\alpha_i$$
  so
  $$\sum_{i \in L}\beta_i >= \frac{2}{3}*\frac{3n}{4}\Delta_{avg} - \frac{1}{6}\sum_{i \in H}\alpha_i$$
  $$\sum_{i \in L}\beta_i >= m - \frac{1}{6}\sum_{i \in H}\alpha_i$$
  And because $\sum_{i \in H}\alpha_i <= 2m$
  We have
  $$\sum_{i \in L}\beta_i >= m - \frac{1}{3}m$$
\end{itemize}
So
$$\sum_{i}\beta_i >= m - \frac{1}{2}\sum_{i \in L}\alpha_i$$
And that way we are able to help enough edges.

\newpage

Let's evaluate the maximal final degree of the nodes
\begin{itemize}
  \item if $i \in L$,
  $$\gamma_i <= 3(2\beta_i +\alpha_i)$$
  and
  $$\beta_i < \frac{2}{3}\Delta_{avg} - \frac{1}{2}\alpha_i + 1$$
  So
  $$\gamma_i < 4\Delta_{avg} + 6$$
  $$\gamma_i <= 4\Delta_{avg} + 5$$
  \item if $i \in HO \cap HI$,
  $$\gamma_i <= 2 + 6\beta_i$$
  if $\beta_i = 0$, $\gamma_i <= 2$
  else
  $$\gamma_i < 2 + 6(\frac{2}{3}\Delta_{avg} - \frac{1}{6}\alpha_i + 1) = 8 + 4\Delta_{avg} - \alpha_i$$
  But $$\alpha_i > 4\Delta_{avg}$$
  So $$\gamma_i < 8$$
  So $$\gamma_i <= 7$$
  \item if $i \in HO \setminus HI$ (same reasoning for $i \in HI \setminus HO$)\\\\
  Let's note $\alpha_i^1$ the initial out-degree of i and $\alpha_i^2$ the initial in-degree
  We have :
  \begin{enumerate}
    \item $\alpha_i = \alpha_i^1 + \alpha_i^2$
    \item $\alpha_i^1 >= 2\Delta_{avg}$
    \item $\alpha_i^2 < 2\Delta_{avg}$
  \end{enumerate}
  $$\gamma_i <= 1 + \alpha_i^2 + 6\beta_i$$
  if $\beta_i = 0$, $\gamma_i <= 2 + \alpha_i^2 < 2 + 2\Delta_{avg}$
  else
  $$\gamma_i < 1 + \alpha_i^2 + 6(\frac{2}{3}\Delta_{avg} - \frac{1}{6}\alpha_i + 1) = 8 + 4\Delta_{avg} - \alpha_i^1$$
  But $$\alpha_i^1 > 2\Delta_{avg}$$
  So $$\gamma_i < 7 + 2\Delta_{avg}$$
  So $$\gamma_i <= 6 + 2\Delta_{avg}$$
  \item if $i \in H \setminus (HI \cup HO)$
  $$\gamma_i <= \alpha_i + 6\beta_i$$
  because $\alpha_i < 4\Delta_i$, $\beta_i = \lceil\frac{2}{3}\Delta_{avg}-\frac{1}{6}\alpha_i\rceil > 0$
  so $$\gamma_i < \alpha_i + 6(\frac{2}{3}\Delta_{avg}-\frac{1}{6}\alpha_i + 1)$$
  $$\gamma_i < 4\Delta_{avg} + 6$$
  $$\gamma_i <= 4\Delta_{avg} + 5$$
\end{itemize}

\end{document}
